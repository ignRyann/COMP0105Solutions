\documentclass[a4paper]{article}[10pt]
\usepackage{setspace}

\pagestyle{plain}
\usepackage[a4paper, margin = 3cm, bottom = 2.5cm]{geometry}
\usepackage{amssymb,graphicx,color}
\usepackage{amsfonts}
\usepackage{latexsym}
\usepackage{amsmath}
\usepackage{hyperref}
\hypersetup{
    colorlinks,
    citecolor=black,
    filecolor=black,
    linkcolor=black,
    urlcolor=black
}
\def \set#1{\{#1\} }

%%%%%%%%%%%%%%%%%%%%%%%%%%


\title{{\vspace{-4em} \includegraphics[scale=0.4]{ucl_logo.png}}\\
{{\vspace{2em} \Huge COMP0105 Past Paper Answers}}\\
{\Large Not Official - Student Made}\\
}
\author{Roman Ryan Karim}

\begin{document}
 
\onehalfspacing
\maketitle
\setcounter{page}{1}

\newpage
\tableofcontents
\newpage

\section{2023 Paper}
\subsection*{Q1}
\addcontentsline{toc}{subsection}{\protect\numberline{}Q1}%

\subsection*{Q2a}
\addcontentsline{toc}{subsection}{\protect\numberline{}Q2a}%
So in total, we have 19 payments. Our first payment is compounded for 16 time periods (of 6 months at 1.015\% per period). Thus, we have that
\begin{equation}
    \text{FV} = (\$200 \times (1.015)^{18}) + (\$200 \times (1.015)^{17}) + .. + (\$200 \times (1.015)^0)
\end{equation}
\begin{equation}
    \text{FV} = \$200 \times \sum^{18}_{i = 0} 1.015^i
\end{equation}

So we have that $a = \$200$, $r = 1.015$ and $n = 19$. This means that we have
\begin{equation}
    FV = \$200 \times \bigg( \frac{1 - 1.015^{19}}{1-1.015} \bigg) = \$4359.34 \text{ (2 d.p.)}
\end{equation}

\subsection*{Q5e}
\addcontentsline{toc}{subsection}{\protect\numberline{}Q5e}
\begin{equation*}
    100 \times \frac{0.015}{0.055} = 27.\dot{2}\dot{7} \approx 27
\end{equation*}

\begin{equation*}
    (100 \times 0.627) - (27 \times 0.537) = 48.201 \approx 48
\end{equation*}

Thus, our portfolio is structured as
\begin{equation*}
    \Pi = 27C_b - 100C_w + 48S
\end{equation*}

Perfect hedging is not possible as we had to round our values and $\Gamma$ changes as the price moves.


\end{document}